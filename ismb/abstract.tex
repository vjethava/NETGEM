%%% abstract.tex --- 
%% 
%% Filename: abstract.tex
%% Description: 
%% Author: Vinay Jethava
%% Maintainer: 
%% Created: Mon Jan  4 17:36:00 2010 (+0530)
%% Version: 
%% Last-Updated: Wed Jan 27 15:21:58 2010 (+0530)
%%           By: Vinay Jethava
%%     Update #: 36
%% URL: 
%% Keywords: 
%% Compatibility: 
%% 
%%%%%%%%%%%%%%%%%%%%%%%%%%%%%%%%%%%%%%%%%%%%%%%%%%%%%%%%%%%%%%%%%%%%%%
%% 
%%% Commentary: 
%% 
%% 
%% 
%%%%%%%%%%%%%%%%%%%%%%%%%%%%%%%%%%%%%%%%%%%%%%%%%%%%%%%%%%%%%%%%%%%%%%
%% 
%%% Change log:
%% 
%% 
%%%%%%%%%%%%%%%%%%%%%%%%%%%%%%%%%%%%%%%%%%%%%%%%%%%%%%%%%%%%%%%%%%%%%%
%% 
%% This program is free software; you can redistribute it and/or
%% modify it under the terms of the GNU General Public License as
%% published by the Free Software Foundation; either version 3, or
%% (at your option) any later version.
%% 
%% This program is distributed in the hope that it will be useful,
%% but WITHOUT ANY WARRANTY; without even the implied warranty of
%% MERCHANTABILITY or FITNESS FOR A PARTICULAR PURPOSE.  See the GNU
%% General Public License for more details.
%% 
%% You should have received a copy of the GNU General Public License
%% along with this program; see the file COPYING.  If not, write to
%% the Free Software Foundation, Inc., 51 Franklin Street, Fifth
%% Floor, Boston, MA 02110-1301, USA.
%% 
%%%%%%%%%%%%%%%%%%%%%%%%%%%%%%%%%%%%%%%%%%%%%%%%%%%%%%%%%%%%%%%%%%%%%%
%% 
%%% Code:
\documentclass{bioinfo}
\usepackage{appendix}
\usepackage{multirow}
% \usepackage{fixltx2e}
% \usepackage{hyperref}
\copyrightyear{2010}
\pubyear{2010}

\begin{document}
\firstpage{1}

\newtheorem{theorem}{Theorem}[section]
\newtheorem{lemma}[theorem]{Lemma}
\newtheorem{proposition}[theorem]{Proposition}
\newtheorem{corollary}[theorem]{Corollary}

\newenvironment{definition}[1][Definition]{\begin{trivlist}
\item[\hskip \labelsep {\bfseries #1}]}{\end{trivlist}}
\newenvironment{example}[1][Example]{\begin{trivlist}
\item[\hskip \labelsep {\bfseries #1}]}{\end{trivlist}}
\newcommand{\todo}[1]{\textcolor{red}{#1}}
\newcommand{\update}[1]{\textcolor{blue}{#1}}
\newcommand{\old}[1]{\textcolor{green}{#1}}
\newenvironment{remark}[1][Remark]{\begin{trivlist}
\item[\hskip \labelsep {\bfseries #1}]}{\end{trivlist}}
\title[NETGEM]{NETGEM: Network Embedded analysis of Temporal Gene Expression using Mixture models}
%\title[short Title]{Analysis of temporal gene expression using mixture models}
\author[Sample \textit{et~al}]{Vinay Jethava$^{1}$, Chiranjib Bhattacharyya$^{1}$, Devdatt Dubhashi$^{2}$,Goutham N. 
Vemuri$^{3}$\footnote{to whom correspondence should be addressed}}
\address{$^{1}$Computer Science and Automation Department, Indian Institute of Science,
Bangalore, INDIA\\
$^{2}$Department of Computer Science, Chalmers University of
  Technology, G\"oteborg, SWEDEN\\
$^{3}$Systems Biology Division, Department of Chemical and Biological Engineering, Chalmers University of
Technology, G\"oteborg, SWEDEN\\
}

\history{Received on XXXXX; revised on XXXXX; accepted on XXXXX}
\editor{Associate Editor: XXXXXXX}

\maketitle


\begin{abstract}
\section{Motivation}
Temporal analysis of gene expression data has been limited to identifying genes whose expression varies with time and/or correlation between genes what have similar temporal profiles. Often, the methods do not consider the underlying network constraints that connect the genes. In addition to identifying changes in the genes, it is becoming increasingly evident that interactions change substantially. Thus far, there is no systematic method to relate the temporal changes in gene expression to the dynamics of interactions between them in the context of a regulatory network. The availability of this data opens up possibilities for discovering new mechanisms of regulation and provides valuable insight into identifying time-sensitive interactions. Furthermore, such a framework would also allow for studies on the effect of a genetic perturbation on the dynamics of the interactions.
\section{Results}
We present NETGEM, a tractable model rooted in Markov dynamics,
 for analyzing temporal profiles of genetic expressions arising 
out of known protein interaction networks evolving with unknown dynamics. 
The model treats the interaction strengths as random variables which
are modulated by suitable priors. This approach is necessitated by the
extremely small sample size of the available observations. The model
is amenable to a linear time algorithm for efficient inference. 
 When applied to real data
 NETGEM was successful in  identifying (i) temporal interactions and determining their strength, (ii) functional categories of the actively interacting partners and (iii) dynamics of interactions in perturbed networks. 
\section{Availability:}
The source code for NETGEM is available from http://www.sysbio.se/BioMet

\section{Contact:} goutham@chalmers.se 
\end{abstract}

\end{document}
%%%%%%%%%%%%%%%%%%%%%%%%%%%%%%%%%%%%%%%%%%%%%%%%%%%%%%%%%%%%%%%%%%%%%%
%%% abstract.tex ends here
